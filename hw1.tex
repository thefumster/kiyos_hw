\documentclass[letterpage]{article}
\usepackage{amsmath}

\begin{document}\noindent
1) \textbf{Dimensionally regularized Green function}
The key formula
\begin{equation}
  \nabla^2 \frac{1}{|\mathbf{x} - \mathbf{x'}|} = -4\pi \,\delta^3 (\mathbf{x} - 
  \mathbf{x'})
\end{equation}
is mathematically tricky to prove because of the singularity at the origin.
Various ways to soften that singularity to regularize the mathematics are
employed. Prove this formula using the form
\begin{equation}
  G(\mathbf{x}-\mathbf{x'}) = \lim_{\eta \to 1^-} \frac{1}{|\mathbf{x}-
  \mathbf{x'}|^\eta},
\end{equation}
with $\eta$ approaching unity from below.\\

\noindent
2)\textbf{Potential energy of a spherical ball}
a)Use Gauss' Law to get the electric field inside and outside a spherical
ball of charge $Q$ and radius $R$, with a uniform charge density.\\
b)Calculate the potential energy that is needed to assemble this configuration
by building up the ball bringing infinitesimal spherical shells in from 
$r = \infty$.\\
c)Verify that this potential energy is equal to
\begin{equation}
  W = \frac{\epsilon_0}{2} \int d^3xE^2(x)
\end{equation}\\

\noindent
3)\textbf{Neutral Hydrogen}\\
Jackson 1.5. Please explicitly verify that the total charge is zero.\\

\noindent
4) \textbf{Capacitance}\\
Jackson 1.7. You do not need to estimate the gauges of wires required.
It would be good to review the formulas relating to capacitance that were
covered in your introductory physics course.

\newpage
\noindent
1)
\begin{align*}
  \nabla^2 \frac{1}{|\mathbf{x} - \mathbf{x'}|} &= -4\pi \,\delta^3 (\mathbf{x} - 
  \mathbf{x'})\\
  G(\mathbf{x}-\mathbf{x'}) &= \lim_{\eta \to 1^-} \frac{1}{|\mathbf{x}-
  \mathbf{x'}|^\eta}
\end{align*}
In spherical coordinates, taking $\mathbf{x'} = 0$,
\begin{equation*}
  G(\mathbf{x}) = \lim_{\eta \to 1^-} \frac{1}{r^\eta}
\end{equation*}
\begin{align*}
  \nabla^2 G(\mathbf{x}) &= \lim_{\eta \to 1^-} \nabla^2 \frac{1}{r^\eta}\\
  &= \lim_{\eta \to 1^-} \frac{1}{r^2}\frac{\partial}{\partial r} r^2
  \frac{\partial}{\partial r}r^{-\eta}\\
  &= \lim_{\eta \to 1^-} \frac{1}{r^2} \frac{\partial}{\partial r} (-\eta r^2
  \cdot r^{-\eta -1})\\
  &= \lim_{\eta \to 1^-} \frac{-\eta}{r^2} \frac{\partial}{\partial r} 
  r^{-\eta + 1}\\
  &= \lim_{\eta \to 1^-} \frac{-\eta}{r^2}(1-\eta) r^{-\eta}\\
  &= \lim_{\eta \to 1^-} -\eta(1-\eta)r^{-\eta - 2}\\  
\end{align*}
\begin{equation*}
  \int d^3x -\eta(1 - \eta) r^{-\eta - 2}
\end{equation*}
\begin{align*}
  &= -4\pi \eta(1 - \eta) \int_0^a dr \,r^{-\eta} \qquad a : \text{constant}\\
  &= -4\pi \eta(1 - \eta) r^{-\eta + 1} \cdot \left. \frac{1}{1 - \eta} 
  \right|_0^a\\
  &= -4\pi \eta \cdot \left. r^{1 - \eta} \right| _0^a
\end{align*}\\
\noindent
For $\eta < 1$, $(1 - \eta > 0)$ we get
\begin{equation*}
  = -4\pi \eta \cdot a^{1 - \eta}
\end{equation*}
\noindent
Then,
\begin{equation*}
  \int d^3x \lim_{\eta \to 1^-} - \eta(1-\eta) r^{-\eta-2}
\end{equation*}
\begin{align*}
  &= \lim_{\eta \to 1^-} \int d^3x - \eta(1-\eta)r^{-\eta-2}\\
  &= \lim_{\eta \to 1^-} -4\pi \eta \cdot a^{1-\eta}\\
  &= -4\pi
\end{align*}\\
\noindent
If $\nabla^2G = -4\pi \delta^3 (\textbf{x})$, integrating both sides gives 
$-4\pi$.\\ 
Also,
\begin{equation*}
  \nabla^2G (\textbf{x}) = \lim_{\eta \to 1^-} - \eta(1-\eta) r^{-\eta-2} = 0
\end{equation*} 
unless r = 0.\\

\noindent
Therefore, we conclude
\begin{equation*}
  G(\textbf{x}) = \lim_{\eta \to 1^-} \frac{1}{r^\eta} = \frac{1}{r}
  = \frac{1}{|\textbf{x}|}
\end{equation*}
is the green function.\\

\noindent
Now, shifting the coordinates by $\mathbf{x'}$ gives
\begin{equation*}
  G(\textbf{x} - \textbf{x}') = \frac{1}{|\textbf{x} - \textbf{x}'|}
\end{equation*}


\end{document}